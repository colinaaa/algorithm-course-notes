\chapter{骨牌问题(Tiling Problem)}

\begin{introduction}
    \item 定义
    \item 形状为$2\times1$的骨牌
\end{introduction}

\section{问题定义}\label{sec:tiling-1}
\begin{definition}{骨牌问题}{def:tiling-problem}
    给出大小确定数量不限的骨牌,问能否不重叠的铺满整个平面
\end{definition}

\section{骨牌形状为\texorpdfstring{$2\times1$}{}}\label{sec:tiling-2}
    骨牌形状为$2\times1$(如图\ref{fig:tiling-1}),判断使用该种骨牌能不能不重叠的铺满给定的平面。
    形状为$2\times1$的骨牌的填充问题属于$P$类问题。
    下面是一些用这种骨牌填充的例题。
    \begin{figure}[hbt!]
        \centering
        \begin{tabular}{|c|}
            \hline
            \\ \hline
            \\ \hline
        \end{tabular}
        \caption{形状为$2\times1$的骨牌}
        \label{fig:tiling-1}
    \end{figure}
\subsection{形状为\texorpdfstring{$M\times N$}{}的矩形}\label{subsec:tiling-2-1}
    这种完整的矩形(如图\ref{fig:tiling-2})是简单问题。只需要判断形状中的格子数目($M\times N$)是否为偶数。
    如果格子的数目为偶数,则能填满。
    如果格子的数目为奇数则不能填满。
    \begin{figure}[hbt!]
        \centering
        \begin{tabular}{|c|c|c|c@{\dots}c|c|c|c|}
            \cline{1-3} \cline{6-8}
            & & & & & & & \\ 
            \cline{1-3} \cline{6-8}
            & & & & & & & \\ 
            \cline{1-3} \cline{6-8}
            \multicolumn{3}{c}{}& &\multicolumn{3}{c}{} \\
            \cline{1-3} \cline{6-8}
            & & & & & & & \\ 
            \cline{1-3} \cline{6-8}
            & & & & & & & \\ 
            \cline{1-3} \cline{6-8}
        \end{tabular}
        \caption{$M\times N$的矩形}\label{fig:tiling-2}
    \end{figure}


    如果格子的数目为奇数,显然不能使用$2\times1$的骨牌铺满。而当格子的数目为偶数时,则行或列中必有一个为偶数。显然能用这种骨牌填满。
\subsection{形状为$6\times6$的矩形挖去两个对角}\label{subsec:tiling-2-2}
    \begin{figure}[ht!]
        \begin{subfigure}{0.5\textwidth}
            \centering
            \begin{tabular}{|c|c|c|c|c|c|}
                \cline{2-6}
                \multicolumn{1}{c|}{}& & & & & \\ \hline
                & & & & & \\ \hline
                & & & & & \\ \hline
                & & & & & \\ \hline
                & & & & & \\ \hline
                & & & & &\multicolumn{1}{c}{} \\ 
                \cline{1-5}
            \end{tabular}
            \caption{挖去两个对角的$6\times6$的矩形}\label{fig:tiling-3}
        \end{subfigure}
        \begin{subfigure}{0.5\textwidth}
            \centering
            \begin{tabular}{|c|c|c|c|c|c|}
                \cline{2-6}
                \multicolumn{1}{c|}{}& \cellcolor[rgb]{0,0,0} & & \cellcolor[rgb]{0,0,0} & & \cellcolor[rgb]{0,0,0} \\ \hline
                 \cellcolor[rgb]{0,0,0} & & \cellcolor[rgb]{0,0,0} & & \cellcolor[rgb]{0,0,0} & \\ \hline
                 & \cellcolor[rgb]{0,0,0} & & \cellcolor[rgb]{0,0,0} & & \cellcolor[rgb]{0,0,0} \\ \hline
                 \cellcolor[rgb]{0,0,0} & & \cellcolor[rgb]{0,0,0} & & \cellcolor[rgb]{0,0,0} & \\ \hline
                 & \cellcolor[rgb]{0,0,0} & & \cellcolor[rgb]{0,0,0} & & \cellcolor[rgb]{0,0,0} \\ \hline
                 \cellcolor[rgb]{0,0,0} & & \cellcolor[rgb]{0,0,0} & & \cellcolor[rgb]{0,0,0} &\multicolumn{1}{c}{} \\ 
                \cline{1-5}
            \end{tabular}
            \caption{染色后的矩形}\label{fig:tiling-4}
        \end{subfigure}
        \caption{}
    \end{figure}
    这种情况下(如图\ref{fig:tiling-3}),格子的总数为34个,是个偶数。所以可以尝试通过对格子进行染色的方式判断。选取其中一个格子染成和黑色,
    对黑色上下左右相邻的格子染成白色,对白色上下左右相邻的格子染成黑色(结果如图\ref{fig:tiling-4})。染色是判断能否被填充的一个方法。
    这种方法得到的不可填充的结论是可信的,但可填充的结论却不一定可信。

    统计黑色和白色格子的数目,黑色有18个,白色有16个。而将骨牌染成黑白两色,发现需要被填充的形状中黑色和白色的个数不同。所以这种形状无法被$2\times1$的骨牌填充。
\subsection{一个奇异的形状}\label{subsec:tiling-2-3}
    \begin{figure}[h!]
        \begin{subfigure}{0.5\textwidth}
            \centering
            \begin{tabular}{|c|c|c|c|c|c|c|c|c|c|}
                \cline{2-4} \cline{7-9}
                \multicolumn{1}{c|}{} & & & & \multicolumn{1}{c}{} & \multicolumn{1}{c|}{} & & & & \multicolumn{1}{c}{} \\ \hline
                 & & & & & & & & & \\ \hline
                 & & & & & & & & & \\ \hline
                 & & & & & & & & & \\ \hline
                \multicolumn{1}{c|}{} & & & & \multicolumn{1}{c}{} & \multicolumn{1}{c|}{} & & & & \multicolumn{1}{c}{} \\ 
                \cline{2-4} \cline{7-9}
            \end{tabular}
            \caption{一个例子}\label{fig:tiling-5}    
        \end{subfigure}
        \begin{subfigure}{0.5\textwidth}
            \centering
            \begin{tabular}{|c|c|c|c|c|c|c|c|c|c|}
                \multicolumn{5}{c!{\color[rgb]{1,0,0}\vline}}{} & \multicolumn{5}{c}{} \\
                \cline{2-4} \cline{7-9}
                \multicolumn{1}{c|}{} & \cellcolor[rgb]{0,0,0} & & \cellcolor[rgb]{0,0,0} & \multicolumn{1}{c!{\color[rgb]{1,0,0}\vline}}{} & \multicolumn{1}{c|}{} & & \cellcolor[rgb]{0,0,0} & & \multicolumn{1}{c}{} \\ \hline
                 \cellcolor[rgb]{0,0,0} & & \cellcolor[rgb]{0,0,0} & & \multicolumn{1}{c!{\color[rgb]{1,0,0}\vline}}{\cellcolor[rgb]{0,0,0}}  & & \cellcolor[rgb]{0,0,0} & & \cellcolor[rgb]{0,0,0} & \\ \hline
                 & \cellcolor[rgb]{0,0,0} & & \cellcolor[rgb]{0,0,0} & \multicolumn{1}{c!{\color[rgb]{1,0,0}\vline}}{} & \cellcolor[rgb]{0,0,0} & & \cellcolor[rgb]{0,0,0} & & \cellcolor[rgb]{0,0,0} \\ \hline
                 \cellcolor[rgb]{0,0,0} & & \cellcolor[rgb]{0,0,0} & & \multicolumn{1}{c!{\color[rgb]{1,0,0}\vline}}{\cellcolor[rgb]{0,0,0}} & & \cellcolor[rgb]{0,0,0} & & \cellcolor[rgb]{0,0,0} & \\ \hline
                \multicolumn{1}{c|}{} & \cellcolor[rgb]{0,0,0} & & \cellcolor[rgb]{0,0,0} & \multicolumn{1}{c!{\color[rgb]{1,0,0}\vline}}{} & \multicolumn{1}{c|}{} & & \cellcolor[rgb]{0,0,0} & & \multicolumn{1}{c}{} \\ 
                \cline{2-4} \cline{7-9}
                \multicolumn{5}{c!{\color[rgb]{1,0,0}\vline}}{} & \multicolumn{5}{c}{} 
            \end{tabular}
            \caption{染色后}\label{fig:tiling-6}
        \end{subfigure}
    \end{figure}
    在这种情况下(如图\ref{fig:tiling-5}),使用上述方式染色后(如图\ref{fig:tiling-6})发现黑色与白色的块数相同,均为21块。但这个图形却是不可被填充的。

    现假设骨牌能填充这个形状,那么红线左侧和红线右侧都应该是可以被填充满的。但红线左侧的黑色方块比白色方块多4个,需要从右面借4个白色方块才能使左红线侧的褐色方块与白色方块数目相同。
    但红线右侧在不借出黑色方块的情况下最多借出2个白色方块,不满足需求。因此左边是不可能被填满的。所以整个平面也是不可被填满的。

    这个问题还可以用网络流来求解。详见\pageref{sec:network-flows-tiling}页或第\ref{sec:network-flows-tiling}章
