\chapter{贪心算法}

\begin{introduction}
	\item 区间调度问题
	\item 任务调度问题
	\item 另一个任务调度问题
\end{introduction}


\section{概述}
\begin{definition}{贪心算法}
    在迭代的每一步根据某种规则取局部最优算法称为贪心算法。对同一个问题可能有许多贪心策略,每种策略有不同的规则,得到的结果也可能不同。
    \\贪心算法主要使用交换法证明,即对任意假定的结果,逐步经过交换,每一步都使新的结果不差于原有的结果,最终得到经过选择的策略求出的解。由此证明经该策略得到的解是最优解。
\end{definition}



\section{区间调度问题}
\begin{example}
    给定区间$I_1, I_2, \ldots, I_n$,
	$s_i$为$I_i$开始时间,$f_i$为$I_i$的结束时间($f_i>s_i$),$I_j=[s_j,f_j]$
	\\目标:选出尽量多的区间,使两两互不重叠。
\end{example}

\subsection{算法设计}
\begin{figure}[hbt!]
	\centering
	\begin{subfigure}{.3\textwidth}
		\centering
		\begin{tikzpicture}
			\draw[|-|] (0, 0) -- node [above] {\scriptsize } (1, 0);
			\draw[|-|] (1.2,0) -- node [above] {\scriptsize } (2.2,0);
			\draw[|-|] (0.5 , -0.5) -- node  [above] {\scriptsize } (1.5, -0.5);
		\end{tikzpicture}
		\caption{}\label{fig:counterexample1}
	\end{subfigure}
	\begin{subfigure}{.3\textwidth}
		\centering
		\begin{tikzpicture}
			\draw[|-|] (0, 0) -- node [above] {\scriptsize }  (2.5, 0);
			\draw[|-|] (0.3, -0.5) -- node [above] {\scriptsize } (0.7, -0.5);
			\draw[|-|] (0.8, -0.5) -- node [above] {\scriptsize } (1.5, -0.5);
			\draw[|-|] (1.6, -0.5) -- node [above] {\scriptsize } (2.0, -0.5);
		\end{tikzpicture}
		\caption{}\label{fig:counterexample2}
	\end{subfigure}
	\begin{subfigure}{.3\textwidth}
		\centering
		\begin{tikzpicture}
			\draw[|-|] (0, 0) -- node [above] {\scriptsize }  (1, 0);
			\draw[|-|] (1.2, 0) -- node [above] {\scriptsize }  (2.2, 0);
			\draw[|-|] (2.4, 0) -- node [above] {\scriptsize }  (3.4, 0);
			\draw[|-|] (3.6, 0) -- node [above] {\scriptsize }  (4.4, 0);
			\draw[|-|] (0.6, -0.5) -- node [above] {\scriptsize } (1.6, -0.5);
			\draw[|-|] (0.6, -1) -- node [above] {\scriptsize } (1.6, -1);
			\draw[|-|] (0.6, -1.5) -- node [above] {\scriptsize } (1.6, -1.5);
			\draw[|-|] (1.8 , -0.5) -- node [above] {\scriptsize } (2.8, -0.5);
			\draw[|-|] (3.0 , -0.5) -- node [above] {\scriptsize } (4.0, -0.5);
			\draw[|-|] (3.0 , -1) -- node [above] {\scriptsize } (4.0, -1);
			\draw[|-|] (3.0 , -1.5) -- node [above] {\scriptsize } (4.0, -1.5);
		\end{tikzpicture}
		\caption{}\label{fig:counterexample3}
	\end{subfigure}
	\caption{贪心反例}\label{fig:counterexample}
\end{figure}

\paragraph*{反例}
在本题中,我们可以选择的贪心策略有很多,以下三种均是贪心策略,但是并不能取得最优结果,说明如下。

\begin{enumerate}
	\item 以时间最短排序,反例如\autoref{fig:counterexample1}
	\item 以开始时间最早排序,反例如\autoref{fig:counterexample2}
	\item 选择冲突最少的区间,反例如\autoref{fig:counterexample3}
\end{enumerate}



\begin{figure}[hbt!]
		\centering
		\begin{tikzpicture}
			\draw[|-|] (0, 0) -- node [above] {\scriptsize 3}  (3, 0);
			\draw[|-|] (5, 0) -- node [above] {\scriptsize 6}  (6, 0);
			\draw[|-|] (0.3, -0.5) -- node [above] {\scriptsize 1 }  (1.3, -0.5);
			\draw[|-|] (1.5, -0.5) -- node [above] {\scriptsize 2 }  (2.5, -0.5);
			\draw[|-|] (3.1, -0.5) -- node [above] {\scriptsize 7 } (6.5, -0.5);
			\draw[|-|] (0.9, -1) -- node [above] {\scriptsize 4 } (3.6, -1);
			\draw[|-|] (3.8, -1) -- node [above] {\scriptsize 5 } (4.4, -1);
			\draw[|-|] (5.6 , -1) -- node [above] {\scriptsize 8 } (7, -1);
		\end{tikzpicture}
	\caption{任务区间}\label{fig:example22}
\end{figure}
\newpage
因此,贪心策略的选取并不是随便的,而是需要依据对应的情况,例如\autoref{fig:example22},为了有更多的区间能被选中,从前往后看,每执行一个任务,我们需要使能够使用时间的开始时间尽量早,才能选取尽量好的下一个任务。该贪心策略说明如下。

令$Q=\{I_1, I_2, \ldots, I_n\},R=\varnothing$

\begin{enumerate}
    \item 对$Q$中$I_1, I_2, \ldots, I_n$以结束时间由小到大排序
    \item 每次将Q中有最小结束时间且不与R中区间冲突的区间$I_i$加入R,将$I_i$从Q中移除,循环往复,直到无法加入为止
\end{enumerate}

\subsection{算法分析}
\paragraph*{正确性证明}
\begin{itemize}
    \item 假定有一个更好的解$OPT'$,对其中的每一个区间进行排序,为$J_1‘,J_2',\ldots,J_n'$,而上面给出的贪心解为$J_1,J_2,\ldots,J_n$。
    \item 从$i=1$开始置换将$OPT'$中的$J_i'置换为J_i$根据我们贪心策略的定义,易得$OPT'$中的区间依然不会冲突而且区间数维持不变。
    \item 完成$OPT'$区间最后一个区间的置换后可知贪心解$OPT$的区间数$\geq OPT'$解的区间数
\end{itemize}

\paragraph*{复杂度}
该算法的时间复杂度即为排序所用时间的复杂度$O(nlogn)$,空间复杂度为$O(n)$。

\section{任务调度问题}
\begin{example}
	给定$n$个任务,对于第$i$个任务,给定完成该任务所需要的时间$t_i$和
	该任务的截止时间$d_i$,总体开始时间$S=0$,要求找出$\forall$任务$i$,
	给定一个时间段$[s_i,f_i]$,其中$f_i - s_i = t_i$,且$s_i\geqslant s$

	\begin{equation}
		s.t. \forall \space l_i = \begin{cases}
			f_i - d_i & f_i > d_i    \\
			0        & f_i \leqslant d_i
		\end{cases}
	\end{equation}
	\\目标:令$L =\max(l_i)$,算法给出的L最小。
\end{example}

\subsection{算法设计}
\paragraph*{贪心策略选择}
为了使最大子任务超时时间最短,一个比较自然的贪心思路是让截止时间早的任务优先执行。

\begin{remark}
    假设对任意的任务截止时间$d_i \neq d_j$
\end{remark}

\subsection{算法分析}

\paragraph*{贪心策略正确性证明}
\begin{enumerate}
    \item 假定存在一个其他策略产生的结果$OPT'$结果比我们的贪心策略结果$OPT$更好。
    \item 那么$OPT'$中的任务序列必然有逆序,否则会与$OPT$结果相同,假设该逆序组合任务序号为$i$与$i+1$,其中$f_i>f_{i+1}$。
    \item 我们现在把它调成顺序,即对他们的位置做交换,显然,这并不会影响他们前后的超时时间。假定他们的开始时间为$S$,消耗时间分别为$t_i$和$t_{i+1}$。对于$i$任务,之前的超时时间为$S+t_i-d_i$,之后的超时时间为$S+t_i+t_{i+1}-d_i$。对于$i+1$号任务,之前的超时时间为$S+t_i+t_{i+1}-d_{i+1}$,之后的超时时间为$S+t_{i+1}-d_{i+1}$。
    \item 根据我们的假设$d_{i+1}<d_i$。易得$S+t_i+t_{i+1}-d_i<S+t_i+t_{i+1}-d_{i+1}$且$S+t_{i+1}-d_{i+1}<S+t_i-d_i$。即调换后$i+1$号任务的超时时间小于调换前$i$号任务的超时时间,调换后$i$号任务的超时时间小于调换前$i+1$号任务的超时时间。那么这两个任务的超时时间的最大值一定比原来的小,而其他任务的超时时间并未改变,也就是经过这样的变换后得到的新解$OPT'' \leq OPT'$。
    \item 对$OPT'$的序列进行冒泡排序可以得到经过我们贪心策略得到的$OPT$,经过4的推导,我们能得出$OPT \leq OPT'$。故我们的策略正确。
\end{enumerate}

\paragraph*{复杂度}
本算法仍旧是一个对序列某个性质进行排序的算法,时间复杂度为O($nlogn$)。空间复杂度为$O(n)$。

\section{另一个任调度问题}
\begin{example}
    有$n$个任务,每个任务有它的权重$w_i$,以及完成任务需要的时间$t_i$。定义任务$i$的完成时间为$c_i$。要求设置任务的执行顺序使$M = \sum_{i=1}^n {w_ic_i}$最小。
\end{example}

\subsection{算法设计}

\paragraph*{贪心策略的选择}
这道题要贪心的东西有点类似任务的密度,即$\frac{w_i}{t_i}$。一个自然的想法是将$\frac{w_i}{t_i}$大的排在前面。

\subsection{算法分析}

\paragraph*{贪心策略正确性的证明}
本例同前一个任务调度的基本思路相同,即证明将逆序的组合($\frac{w_i}{t_i} < \frac{w_{i+1}}{t_{i+1}}$)换成顺序的组合后$M$变小了。

\begin{enumerate}
    \item 对逆序$(i,i+1)$,假设开始时间为$T_0$,仅对这两项计算代价(调换他们的顺序并不会对前后产生影响)。定义对每一项任务而言代价为$w_ic_i$。
    \item 进过简单的计算,调换前的代价为$x$,调换之后的代价为$y$。
    \begin{equation}
        x=(T_0+t_i)w_i+(T_0+t_i+t_{i+1})w_{i+1}
    \end{equation}
    \begin{equation}
        y=(T_0+t_{i+1})w_{i+1}+(T_0+t_i+t_{i+1})w_i
    \end{equation}
    \begin{equation}
        x-y=t_iw_{i+1}-t_{i+1}w_i
    \end{equation}
    \item 我们已知$\frac{w_i}{t_i} < \frac{w_{i+1}}{t_{i+1}}$,由于这四个数都大于0,所以$2.4$式显然成立,故调换顺序后代价小于于原逆序代价。贪心策略正确。
\end{enumerate}

\paragraph*{复杂度}
本算法仍旧是一个对序列某个性质进行排序的算法,时间复杂度为O($nlogn$)。空间复杂度为$O(n)$。
