\chapter{分治法之大数乘法}
\begin{introduction}
    \item 问题背景
    \item 算法1
    \item 算法2
    \item 伪代码及复杂度分析
\end{introduction}

\section{问题描述}
给定两个大数$A$和$B$, 试计算
\begin{math}
    A \times B
\end{math}.
其中$A$和$B$分别表示为
\begin{math}
    A = a_n a_{n-1} a_{n-2} ... a_2 a_1
\end{math}
,
\begin{math}
    B = b_n b_{n-1} b_{n-2} ... b_2 b_1
\end{math}.

\begin{lemma}{“$A + B$”的复杂度}{label_for_a+b}
    计算$A + B$,其复杂度为$O(n)$, 其中$n$为$A$和$B$的十进制位数。
\end{lemma}
\begin{theorem}{“$A \times B$”的朴素算法复杂度}{label_for_a*b}
    朴素算法下,将$A$与$B$的各位相乘,再相加各次相乘的结果。
    不难得出,这一过程需要进行$n$次基本乘法与$n+1$次加法。
    根据引理\ref{lem:label_for_a+b},朴素算法下计算$A \times B$的时间复杂度为$O(n^2)$.
\end{theorem}
由定理\ref{thm:label_for_a*b}和引理\ref{lem:label_for_a+b}可知,朴素的大数乘法相比加法运算在时间复杂度上高出了一个量级。
由于乘法在计算机中大量存在,我们希望找到更好的算法来降低乘法计算的时间复杂度,以提升计算机的性能。
分治法为我们提供了一条途径。
\section{直接分治法}
\subsection{算法描述}
这是一种简单的分治方法,将两个大数分为前后两部分,进行相乘。不失一般性,这里假设$n$为偶数。将$A$与$B$分割为$A_2$,$A_1$,$B_2$,$B_1$,即:
\begin{displaymath}
    A_2 = a_{n} a_{n-1} ... a_{\frac{n}{2} + 2} a_{\frac{n}{2} + 1}
\end{displaymath}
\begin{displaymath}
    A_1 = a_{\frac{n}{2}} a_{\frac{n}{2} - 1} ... a_2 a_1
\end{displaymath}
\begin{displaymath}
    B_2 = b_{n} b_{n-1} ... b_{\frac{n}{2} + 2} b_{\frac{n}{2} + 1}
\end{displaymath}
\begin{displaymath}
    B_1 = b_{\frac{n}{2}} b_{\frac{n}{2} - 1} ... b_2 b_1
\end{displaymath}\\
则$A$可以写为$A = A_2 \times 2^{\frac{n}{2}} + A_1$.
$B$可以写为$B = B_2 \times 2^{\frac{n}{2}} + B_1$.\\
计算$A \times B$的问题在进行上述转换后表示为:
\begin{displaymath}
    \begin{split}
        A \times B
        & = (A_2 \times 2^{\frac{n}{2}} + A_1) \times (B_2 \times 2^{\frac{n}{2}} + B_1) \\
        & = A_2 B_2 \times 2^n + (A_2 B_1 + A_1 B_2) \times 2^{\frac{n}{2}} + A_1 B_1
    \end{split}
\end{displaymath}
此时将两个大数相乘的问题转化为4个乘法子问题和3个加法子问题。显然,分支策略还可以对子问题使用,继续减小问题的规模。
\subsection{伪代码}
\begin{algorithm}
    \DontPrintSemicolon
    \KwIn{Two large numbers $A$, $B$, which both have $n$ decimal digits}
    \KwResult{$A \times B$}
    \Begin{
        $n \leftarrow $Number of Decimal Digits of $A$ and $B$\;
        \If{$n \neq 1$}{
            Divide $A$, $B$ into $A_2$, $A_1$, $B_2$ and $B_1$\;
            \KwRet{$DirectDAC(A_2, B_2) \ll n + (DirectDAC(A_2, B_1) + DirectDAC(A_1, B_2)) \ll (n \gg 1) + DirectDAC(A_1, B_1)$}\;
        }
        \Else{
            \KwRet{$A \times B$}
        }
    }
    \caption{DirectDAC\label{IR}}
\end{algorithm}
\subsection{复杂度分析}
