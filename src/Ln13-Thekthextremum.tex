\chapter{分治算法之第 K 极值}
\begin{introduction}
	\item 问题引入
	\item 分治思路
	\item 复杂度分析
	\item 优化
	\item 其他值得关注的点
	\item 分治思路
	
\end{introduction}
\section{问题引入}
给定一个长度为 $n$的序列 $N$,求整个序列中第 $k$小的数。
\section{分治思路}
称这个问题为$\text { question }(N, k)$,模仿快速排序的做法,在序列中随机选择一个数作为中间标记tag后将
小于tag数放到tag左边,其余数放到tag右边。
\begin{figure}[h]
	\begin{minipage}[t]{1\linewidth}
		\centering
		\includegraphics[width=10cm,height=3.5cm]{image/kth1.png}
		\caption{模仿快速排序}
	\end{minipage}
\end{figure}
这样我们得到了两个小一点的序列。为了后文的叙述方便,我们称在  t a g  左 侧的序列为  $N_{l}$,  长度为  $\left|N_{l}\right|$,  在$ \operatorname{tag}$ 右侧的序列为 $N_{r}$,  长度为$  \left|N_{r}\right|$ 。
到这一步,我们将问题拆解为如下三种情况:
$$
\text {question}(N, k)=\left\{\begin{aligned}
\text {question}\left(N_{l}, k\right) ,&\left|N_{l}\right|>=k \\
\text {tag} ,&\left|N_{l}\right|=k-1 \\
\text {question}\left(N_{r}, k-\left|N_{l}\right|-1\right) ,&\left|N_{l}\right|<k-1
\end{aligned}\right.
$$
按照上述方法不断递归下去,直到中间某一步达成 $ \left|N_{l}\right|=k-1$  的条件时, 就得到想要的结果了。
