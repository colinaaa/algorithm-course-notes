\chapter{分治算法之第 K 极值}
\begin{introduction}
	\item 问题引入
	\item 分治思路
	\item 复杂度分析
	\item 优化
	\item 其他值得关注的点
	\item 分治思路
	
\end{introduction}
\section{问题引入}
给定一个长度为 $n$的序列 $N$,求整个序列中第 $k$小的数。
\section{分治思路}
称这个问题为$\text { question }(N, k)$,模仿快速排序的做法,在序列中随机选择一个数作为中间标记tag后将
小于tag数放到tag左边,其余数放到tag右边。
\begin{figure}[h]
	\begin{minipage}[t]{1\linewidth}
		\centering
		\includegraphics[width=10cm,height=3.5cm]{image/kth1.png}
		\caption{模仿快速排序}
	\end{minipage}
\end{figure}
这样我们得到了两个小一点的序列。为了后文的叙述方便,我们称在  t a g  左 侧的序列为  $N_{l}$,  长度为  $\left|N_{l}\right|$,  在$ \operatorname{tag}$ 右侧的序列为 $N_{r}$,  长度为$  \left|N_{r}\right|$ 。
到这一步,我们将问题拆解为如下三种情况:
$$
\text {question}(N, k)=\left\{\begin{aligned}
\text {question}\left(N_{l}, k\right) ,&\left|N_{l}\right|>=k \\
\text {tag} ,&\left|N_{l}\right|=k-1 \\
\text {question}\left(N_{r}, k-\left|N_{l}\right|-1\right) ,&\left|N_{l}\right|<k-1
\end{aligned}\right.
$$
按照上述方法不断递归下去,直到中间某一步达成 $ \left|N_{l}\right|=k-1$  的条件时, 就得到想要的结果了。
\section{复杂度分析}
因为算法中每一步的 tag都是随机取的,所以我们在分析复杂度时不仅要 考虑理想情况,也要考虑最坏情况。
最佳情况  \quad  每一次的  t a g  都恰好为序列的中间值,即每次操作都将问题规 模缩小为原来的一半。于是由经典的分治算法复杂度计算我们可以得到:
$$T(n)=T\left(\frac{n}{2}\right)+n$$
由主定理可以得到此时的时间复杂度
$$T(n) \in O(n)$$

最坏情况 每一次tag都恰好为序列的最大值或者最小值,即每次操作都 只能将问题规模减 1。于是我们可以的得到:
$$
T(n)=1+2+3+\cdots+(n-1)+n
$$
即
$$
T(n)=\frac{n(n+1)}{2}
$$
所以最坏情况的时间复杂度
$$
T(n) \in O\left(n^{2}\right)
$$
中间情况  \quad  我们虽然很难保证每次  t a g  的选取都恰好选中序列的中间值,但
如果我们能保证每次选取的  t a g  都在某个区间范围内,这种中间情况会比 最坏情况好很多。
比如,假如我们能够保证每次操作得到的  
$$\left|N_{l}\right|  $$
都满足式:

$$\frac{1}{4}|N| \leq\left|N_{l}\right| \leq \frac{3}{4}|N|$$

这样每次操作至少将问题规模缩小为原来的  $$\frac{3}{4} \circ $$ 于是有:

$$T(n) \leq T\left(\frac{3}{4} n\right)+n$$

由主定理得:

$$T(n) \in O(n)$$

\section{优化}
在复杂度分析-最坏情况中我们提到的算法有着明显的缺陷,即在最坏情况 下,算法的时间复杂度可能是 
$$O\left(n^{2}\right)$$
级别的。但是在复杂度分析-中间情 况中,我们也提到了一种中间情况,在这种情况下,算法的时间复杂度和最 优情况的时间复杂度是近似的,都是  O(n)  。
那么,我们该如何选取  t a g,  以保证式 成立呢?
两次取中法
1. 将 $ \mathrm{n}$  个元素每 5 个一组,分成  $\frac{n}{5}$(  上界  )  组。
2. 取出每一组的中位数。
3. 查找上一步中所有中位数的中位数作为 tag,偶数个中位数的情况下 设定为选取中间较小的一个。
直观上如下图所示:
\begin{figure}[h]
	\begin{minipage}[t]{1\linewidth}
		\centering
		\includegraphics[width=10cm,height=3.5cm]{image/kth2.png}
	\end{minipage}
\end{figure}
这样一来,X部分的数一定都大于mm, 而W  部分的数一定都小于  m m  。 所以,每次操作至少有  $$\frac{3 n-6}{10} $$ 个元素被易除,即最多只剩  $\frac{7 n+6}{10} $ 个元素。
第一次组内取中时,每组消耗常数时间,共有  $ \frac{n}{5} $  组,因此第一次取中的时 间复杂度为  $ c n \in O(n) $ ( c  为某常数  ) 。  第二次中位数序列取中,实质上等价 于问题 
$\text { question }\left(N_{m}, \frac{\left|N_{m}\right|}{2}\right)$,
($N_{m}$为中位数序列)而  $ \left|N_{m}\right|=\frac{n}{5} $ ,  所以, 可 以写出优化后的算法时间复杂度递推式:


$$T(n) \leq T\left(\frac{n}{5}\right)+T\left(\frac{7 n+6}{10}\right)+O(n)$$
接下来我们将证明
$$T(n) \in O(n)$$
首先,根据  T(n)  的定义,  T(n)  是一个单调递增函数。当  n  足够大时,我 们可以对式 进行放缩,得到式
$$T(n) \leq T\left(\frac{n}{5}\right)+T\left(\frac{3}{4} n\right)+O(n)$$
根据 $ O(n) $ 的定义,式还 可写成
$$T(n) \leq T\left(\frac{n}{5}\right)+T\left(\frac{3}{4} n\right)+c n$$
其中  c  为某常数。接下来我们试图证明
$$T(n) \leq 20 \mathrm{cn}$$
我们知道 $ n=1 $ 时有
$$T(n) \leq 20 \mathrm{cn} \quad(n=1)$$
假设 $ n \leq k$  时式 也成立,那么当  $n=k+1$  时,有
$$
\frac{n}{5} \leq k \\
\frac{3}{4} n \leq k
$$


